% Options for packages loaded elsewhere
\PassOptionsToPackage{unicode}{hyperref}
\PassOptionsToPackage{hyphens}{url}
%
\documentclass[
  openany]{book}
\usepackage{amsmath,amssymb}
\usepackage{lmodern}
\usepackage{ifxetex,ifluatex}
\ifnum 0\ifxetex 1\fi\ifluatex 1\fi=0 % if pdftex
  \usepackage[T1]{fontenc}
  \usepackage[utf8]{inputenc}
  \usepackage{textcomp} % provide euro and other symbols
\else % if luatex or xetex
  \usepackage{unicode-math}
  \defaultfontfeatures{Scale=MatchLowercase}
  \defaultfontfeatures[\rmfamily]{Ligatures=TeX,Scale=1}
\fi
% Use upquote if available, for straight quotes in verbatim environments
\IfFileExists{upquote.sty}{\usepackage{upquote}}{}
\IfFileExists{microtype.sty}{% use microtype if available
  \usepackage[]{microtype}
  \UseMicrotypeSet[protrusion]{basicmath} % disable protrusion for tt fonts
}{}
\makeatletter
\@ifundefined{KOMAClassName}{% if non-KOMA class
  \IfFileExists{parskip.sty}{%
    \usepackage{parskip}
  }{% else
    \setlength{\parindent}{0pt}
    \setlength{\parskip}{6pt plus 2pt minus 1pt}}
}{% if KOMA class
  \KOMAoptions{parskip=half}}
\makeatother
\usepackage{fancyvrb}
\usepackage{xcolor}
\IfFileExists{xurl.sty}{\usepackage{xurl}}{} % add URL line breaks if available
\IfFileExists{bookmark.sty}{\usepackage{bookmark}}{\usepackage{hyperref}}
\hypersetup{
  pdftitle={ENCUESTA DE VIAJES Y TURISMO DE LOS HOGARES},
  pdfauthor={Dirección Nacional de Mercados y Estadística - Subsecretaría de Desarrollo Estratégico},
  hidelinks,
  pdfcreator={LaTeX via pandoc}}
\urlstyle{same} % disable monospaced font for URLs
\VerbatimFootnotes % allow verbatim text in footnotes
\usepackage{longtable,booktabs,array}
\usepackage{calc} % for calculating minipage widths
% Correct order of tables after \paragraph or \subparagraph
\usepackage{etoolbox}
\makeatletter
\patchcmd\longtable{\par}{\if@noskipsec\mbox{}\fi\par}{}{}
\makeatother
% Allow footnotes in longtable head/foot
\IfFileExists{footnotehyper.sty}{\usepackage{footnotehyper}}{\usepackage{footnote}}
\makesavenoteenv{longtable}
\usepackage{graphicx}
\makeatletter
\def\maxwidth{\ifdim\Gin@nat@width>\linewidth\linewidth\else\Gin@nat@width\fi}
\def\maxheight{\ifdim\Gin@nat@height>\textheight\textheight\else\Gin@nat@height\fi}
\makeatother
% Scale images if necessary, so that they will not overflow the page
% margins by default, and it is still possible to overwrite the defaults
% using explicit options in \includegraphics[width, height, ...]{}
\setkeys{Gin}{width=\maxwidth,height=\maxheight,keepaspectratio}
% Set default figure placement to htbp
\makeatletter
\def\fps@figure{htbp}
\makeatother
\setlength{\emergencystretch}{3em} % prevent overfull lines
\providecommand{\tightlist}{%
  \setlength{\itemsep}{0pt}\setlength{\parskip}{0pt}}
\setcounter{secnumdepth}{5}
  %%% REFERENCIAS
        \usepackage{hyperref}
        % links del indice en negro; citas y URL en azul
        \hypersetup{colorlinks = true, urlcolor={blue}, citecolor={blue}, linkcolor ={blue}}
\usepackage[spanish]{babel} % Idiomas en los que se escribe el documento. 
\usepackage{booktabs}
\usepackage{amsthm}

\usepackage[final]{pdfpages}

\makeatletter
\def\thm@space@setup{%
  \thm@preskip=8pt plus 2pt minus 4pt
  \thm@postskip=\thm@preskip
}
\makeatother
\let\oldmaketitle\maketitle
\AtBeginDocument{\let\maketitle\relax}
\ifluatex
  \usepackage{selnolig}  % disable illegal ligatures
\fi
\usepackage[]{natbib}
\bibliographystyle{apalike}

\title{ENCUESTA DE VIAJES Y TURISMO DE LOS HOGARES}
\usepackage{etoolbox}
\makeatletter
\providecommand{\subtitle}[1]{% add subtitle to \maketitle
  \apptocmd{\@title}{\par {\large #1 \par}}{}{}
}
\makeatother
\subtitle{INFORME METODOLÓGICO INICIAL}
\author{Dirección Nacional de Mercados y Estadística - Subsecretaría de Desarrollo Estratégico}
\date{18 de October de 2021}

\begin{document}
\maketitle

\includepdf[pages={1}, scale=1]{DT1Portada.pdf}
\newpage

\let\maketitle\oldmaketitle
\maketitle

{
\setcounter{tocdepth}{1}
\tableofcontents
}
\hypertarget{presentaciuxf3n}{%
\chapter*{Presentación}\label{presentaciuxf3n}}
\addcontentsline{toc}{chapter}{Presentación}

\hypertarget{documento-tuxe9cnico-nuxba1---resumen}{%
\subsection*{Documento Técnico Nº1 - Resumen}\label{documento-tuxe9cnico-nuxba1---resumen}}
\addcontentsline{toc}{subsection}{Documento Técnico Nº1 - Resumen}

\hypertarget{flujo-turistico}{%
\chapter{\texorpdfstring{\textbf{PRESENTACIÓN Y OBJETIVOS}}{PRESENTACIÓN Y OBJETIVOS}}\label{flujo-turistico}}

\hypertarget{introducciuxf3n-objetivo-del-documento}{%
\section{\texorpdfstring{\textbf{INTRODUCCIÓN: OBJETIVO DEL DOCUMENTO}}{INTRODUCCIÓN: OBJETIVO DEL DOCUMENTO}}\label{introducciuxf3n-objetivo-del-documento}}

Este informe metodológico tiene como fin presentar las tareas, técnicas
y métodos que MacroConsulting utilizará para dar cumplimiento a lo
solicitado en El Pliego de Bases y Condiciones Particulares (en adelante
El Pliego) y en la propuesta de trabajo de MacroConsulting presentada
durante el Proceso Licitatorio (en adelante La Propuesta), estipulado
por el Concurso Público Nacional de Etapa Única No 28-0004-CPU19, con el
propósito de realizar la Encuesta de Viajes y Turismo (en adelante
EVyTH)\footnote{El presente informe se enmarca dentro de los
  requerimientos exigidos por El Pliego (ver página 33).}. Para el relevamiento y posterior estudio de los
viajes de los argentinos se hará uso de técnicas mundialmente aceptadas
y se considerará la experiencia acumulada en nuestro país. La propuesta
de trabajo sobre la que se propone llevar adelante la EVyTH, se
estructura sobre las siguientes pautas básicas:

\begin{enumerate}
\def\labelenumi{\arabic{enumi})}
\tightlist
\item
  Diseño muestral en el cual se pautará el marco a utilizar, unidad de
  análisis, manejo de reemplazos, tamaño muestral, período de
  encuestamiento, etc.
\item
  Procedimiento en el campo, meses de referencia, etc.
\item
  Instrumento tecnológico para la recolección de datos, consistencia,
  almacenamiento, etc.
\item
  Sistema de Información para seguimiento y constatación en tiempo
  real del campo.
\item
  Metodología de validación e imputación de datos y calibración de la
  muestra.
\item
  Análisis de resultados en tiempo real y elaboración mensual de
  informes con resultados de ondas (informes de anticipos y finales).
\item
  Especificación y dimensionamiento de equipos de campo, metodológico,
  estadísticos, administrativo y coordinación.
\end{enumerate}

En base a esta enumeración, el informe se ordena de la siguiente forma:
en el apartado a continuación se repasa los objetivos generales y
específicos de la EVyTH y sus antecedentes. Posteriormente se detallan
los aspectos metodológicos y técnicos sobre los que se estructura el
relevamiento encomendado. Luego, se presentan los resultados esperados
de la consultoría. Finalmente, en anexo se adjuntan algunos detalles
adicionales de la propuesta metodológica (descripción de la estructura
del Formulario Base, Metodología de Imputación, Manual del Encuestador,
Cuadernillo de Ejercicios -para el instructor o los encuestadores-,
Curriculums Vitae del personal clave, etc.).

\hypertarget{objetivos-de-la-evythy-sus-antecedentes}{%
\section{\texorpdfstring{\textbf{OBJETIVOS DE LA EVYTHY: SUS ANTECEDENTES}}{OBJETIVOS DE LA EVYTHY: SUS ANTECEDENTES}}\label{objetivos-de-la-evythy-sus-antecedentes}}

Para definir adecuadamente el marco metodológico de la EVyTH deben
considerarse claramente sus objetivos y antecedentes. Con el fin de
delinear correctamente estos aspectos, MacroConsulting se ha basado en
El Pliego, en la propuesta realizada en la oferta presentada por
MacroConsulting durante el proceso licitatorio (en adelante, La Oferta),
en reuniones posteriores mantenidas con la contraparte técnica del
Ministerio de Turismo y Deportes de la Nación (en adelante, Organismo
Contratante), en la experiencia acumulada en la realización de encuestas
similares en Argentina y en la experiencia internacional en
relevamientos análogos.

Además, se ha ido ganando conocimiento y experiencia específica a partir
del desarrollo de las ediciones de la EVyTH del año 2006 (EVyTH-06, en
adelante), de los años 2011, 2012, 2013, 2014, 2015, 2016, 2017, 2018,
2019 y 2020.

\hypertarget{objetivos-de-la-evyth}{%
\subsection{\texorpdfstring{\textbf{OBJETIVOS DE LA EVYTH}}{OBJETIVOS DE LA EVYTH}}\label{objetivos-de-la-evyth}}

De acuerdo con el Pliego, el objetivo es ``realizar la Encuesta de Viajes
y Turismo de los Hogares --EVyTH- durante doce ondas consecutivas, a
razón de una onda por mes calendario, para relevar los viajes turísticos
realizados, sus características, así como también las expectativas de
viajes durante los períodos relevantes (temporada invernal y
estival)''\footnote{Ver hoja 25 de El Pliego.}.

Por lo tanto, se pretende con la EVyTH medir la evolución de los viajes
realizados por los hogares residentes en los grandes aglomerados urbanos
de Argentina en el periodo de referencia, las características de los
recorridos turísticos (lugares visitados, forma de alojamiento,
estadías, utilización de paquete turístico, medios de transporte, etc.),
los viajes a segundas viviendas, el gasto de cada uno de los viajes y
los aspectos socio- demográficos de los hogares viajeros (ingreso total
del hogar, ocupación principal del jefe de hogar, etc.).

Además, la EVyTH debe incluir también un estudio prospectivo al relevar
las expectativas de viajes durante los periodos de interés turístico
(viajes de los argentinos para las temporadas invernal y estival).

La telesis general de la EVyTH es complementar la información para
estimar la oferta y demanda de bienes y servicios turísticos a través de
medir el turismo de los residentes en Argentina (tanto Turismo Nacional
como Turismo Interno). A partir de esta encuesta se espera obtener
información desagregada del comportamiento turístico de los hogares de
nuestro país y así permitir el dimensionamiento, en términos económicos,
del peso del turismo interno en el conjunto del consumo de bienes y
servicios turísticos en la Argentina. Con dicha información se pueden
efectuar estudios analíticos de los diferentes determinantes de la
actividad turística de los residentes en nuestro país y, por tanto,
diseñar e implementar políticas públicas sectoriales con mayor eficacia
y eficiencia. En consecuencia, la información recolectada sirve para:

\begin{enumerate}
\def\labelenumi{\arabic{enumi})}
\tightlist
\item
  Conocer y caracterizar al turismo nacional.
\item
  Como insumo para estimaciones de los gastos de los hogares en viajes
  y turismo, esencialmente para la elaboración de la Cuenta Satélite
  de Turismo.
\item
  Como instrumento de diseño de políticas de incentivo o fomento del
  turismo mediante el estudio de las características sociodemográficas
  de los hogares que viajan y los que no viajan, así como de las
  modalidades de los viajes.
\item
  Medir la repercusión de determinadas políticas de promoción del
  turismo (por ejemplo, estudiar las consecuencias de la diagramación
  de los feriados nacionales).
\end{enumerate}

El objetivo específico de la EVyTH es recolectar información a través de
una encuesta estructurada sobre técnicas mundialmente aceptadas y sobre
la experiencia acumulada en nuestro país. En particular, esta
experiencia se debe complementar con el conocimiento recabado por
MacroConsulting durante:

\begin{enumerate}
\def\labelenumi{\arabic{enumi})}
\tightlist
\item
  la realización de la Prueba Piloto, que sirvió de base para definir
  y testear la metodología de relevamiento de la de la EVyTH,
\item
  la EVyTH llevada a cabo entre febrero y abril de 2011 (en adelante,
  EVyT-11),
\item
  la EVyTH realizada durante el año 2012 (en adelante, EVyTH-12),
\item
  la EVyTH desarrollada durante el año 2013 (en adelante, EVyTH-13),
\item
  la EVyTH desarrollada durante el año 2014 (en adelante, EVyTH-14),
\item
  la EVyTH desarrollada durante el año 2015 (en adelante, EVyTH-15),
\item
  la EVyTH desarrollada durante el año 2016 (en adelante, EVyTH-16),
\item
  la EVyTH desarrollada durante el año 2017 (en adelante, EVyTH-17),
\item
  la EVyTH desarrollada durante el año 2018 (en adelante, EVyTH-18),
\item
  la EVyTH desarrollada durante el año 2019 (en adelante, EVyTH-19),
\item
  la EVyTH desarrollada durante el año 2020 (en adelante, EVyTH-20).
\end{enumerate}

Se anticipan a continuación, en una enumeración no taxativa, la
información que será recabada durante el operativo de campo:

\begin{itemize}
\item
  La evolución de la cantidad de viajes realizados en el período de
  referencia.
\item
  Las características de los recorridos turísticos durante el período
  de referencia (lugares visitados, forma de alojamiento, estadías,
  utilización de paquete turístico, medios de transporte, etc.).
\item
  El uso de segundas viviendas y las características de los viajes
  realizados a ellas durante el período de referencia.
\item
  El gasto de cada uno de los viajes realizados en Argentina y en el
  exterior durante el período de referencia.
\item
  Las características sociodemográficas de las personas que realizan y
  no realizan viajes durante el periodo (edad, sexo, nivel educativo,
  condición de actividad) y de los hogares que ellas integran (lugar
  de residencia, nivel de ingreso, tamaño del hogar, sexo, edad, nivel
  educativo y condición de actividad del jefe).
\item
  Cantidad de personas y características de los viajes que realizarán
  o probablemente realizaron los turistas durante los fines de semana
  largo.
\item
  Flujo turístico esperado para fechas especiales (temporada de
  invierno y verano) en función de las expectativas de viaje.
\end{itemize}

En el Anexo a este documento se detalla el contenido del cuestionario a
implementar, el cual fue previamente validado con el personal técnico
del Organismo Contratante.

\hypertarget{antecedentes-de-la-evyth}{%
\subsection{\texorpdfstring{\textbf{ANTECEDENTES DE LA EVYTH}}{ANTECEDENTES DE LA EVYTH}}\label{antecedentes-de-la-evyth}}

La EVyTH tiene como objetivo fundamental la caracterización de los
viajes que realizan y/o realizarán los miembros de los hogares
residentes en Argentina, para lo cual se releva dónde van, qué medios de
transporte utilizan, dónde se alojan, cuáles son los motivos por los que
viajan, cuánto gastan, qué opinión tienen sobre los lugares, etc. En
base a este objetivo, se pueden mencionar dos tipos de antecedentes que
servirán de fuentes para determinar el mejor diseño metodológico a
utilizar durante la EVyTH-21; a saber: antecedentes locales y
antecedentes internacionales.

\hypertarget{antecedentes-locales-de-la-evyth}{%
\subsubsection{\texorpdfstring{\textbf{ANTECEDENTES LOCALES DE LA EVYTH}}{ANTECEDENTES LOCALES DE LA EVYTH}}\label{antecedentes-locales-de-la-evyth}}

Consecuentemente, con la importancia que el turismo adquirió en las
últimas décadas en términos sociales, culturales y económicos, diversos
países han procurado obtener información que permitiese mensurar y
caracterizar el fenómeno. Una de las herramientas fundamentales para
este objetivo lo constituyen las investigaciones cuantitativas
implementadas a partir de encuestas. Tal como ha sucedido en el caso
argentino desde inicios de este siglo, las primeras experiencias
estuvieron dirigidas a captar el turismo internacional\footnote{Por ejemplo, la Encuesta de Turismo Internacional (ETI)
  que actualmente lleva adelante el Organismo Contratante
  conjuntamente con el INDEC.} y a
mensurar la llegada de viajeros a las plazas hoteleras de las ciudades
más importantes y de los principales destinos turísticos\footnote{En Argentina existe la Encuesta de Ocupación Hotelera
  (relevamiento realizado conjuntamente entre el Organismo Contratante
  y el INDEC).}.

Sin embargo, este tipo de investigaciones dejan abierta una importante
brecha en lo que refiere al conocimiento del comportamiento turístico de
las personas y su impacto socioeconómico, en particular en lo referido
al turismo interno, es decir, los viajes que los residentes de un país
realizan dentro de él, fundamentalmente viajes en los que el alojamiento
utilizado no es un establecimiento hotelero. Además, en estos
relevamientos es escasa la información relevada sobre las
características de los viajeros.

Primero a partir de módulos especiales en las encuestas de gastos o
presupuestos de los hogares (con el fin primordial de medir el gasto de
los hogares en turismo) y luego a partir de encuestas específicas, en
los últimos años han tenido lugar diversas experiencias cuyo objetivo
radica en cuantificar y caracterizar los viajes realizados, a partir de
encuestas a hogares.

En el caso argentino, la primera experiencia de este tipo fue realizada
en conjunto por el INDEC (Instituto Nacional de Estadística y Censos) y
la por entonces Secretaría de Turismo de la Nación. Dicha investigación,
la EVyTH-06, prácticamente sin antecedentes en la región, conjugaba las
ricas y extensas experiencias internacionales en la materia, en especial
de la FAMILITUR española, conjuntamente con la sólida trayectoria
técnica en la realización de encuestas a hogares del INDEC. La EVyTH-06
se configuró como una encuesta proyectada para dar cuenta, bajo las
recomendaciones técnicas de la Organización Mundial del Turismo, de las
particularidades de los viajes y el turismo que los residentes en el
país realizan.

La prueba piloto de la EVyTH-06 se llevó a cabo en 2005, y el
relevamiento de la encuesta definitiva se realizó entre fines de 2006 e
inicios de 2007, en 10.000 hogares residentes en las capitales de cada
una de las provincias y en los aglomerados que, sin ser capitales,
tuvieran 100.000 habitantes o más.

La segunda experiencia fue recabada a fines del año 2010. Este
relevamiento, denominado Prueba Piloto para la Encuesta de Turismo
Doméstico en Hogares Residentes, permitió la fijación de las bases
metodológicas para llevar adelante una nueva encuesta que permitiese
medir la evolución de los viajes realizados por los hogares residentes y
sus particularidades. La mencionada encuesta fue puesta en marcha en
junio de 2011 (luego de su correspondiente Prueba Piloto) y se concentró
en medir el período estival de ese mismo año (enero a marzo de 2011). Le
siguió un período de relevamiento adicional para computar los viajes y
sus características durante el periodo de abril. Posteriormente,
MacroConsulting articuló una nueva EVyTH para el periodo 2011/2012.
Aquí, además de relevar y caracterizar los viajes realizados durante los
dos meses calendario anterior al mes de relevamiento, se propuso agregar
el relevamiento de expectativas de viaje para fechas especiales y
realizar análisis adicionales. Por ejemplo, se realizó un estudio sobre
las características de los viajes con más de una etapa; también se
estudiaron los destinos turísticos más visitados entre enero y
septiembre de 2012, que sirvió como primera aproximación a la
distribución geográfica de los turistas argentinos; y finalmente, se
implementaron estudios particulares de las temporadas invernales y de
verano. Por último, durante los años 2013 a 2016, MacroConsulting llevó
a cabo la EVyTH, confeccionando informes de relevancia sustantiva para
la caracterización del turismo nacional como lo fueron el informe de
viajes multidestinos, el estudio sobre destinos turísticos, informe de
resultados anuales, entre otros.

Consecuentemente con estos antecedentes, el marco metodológico que se
propone seguir para esta versión ``2021'' de la EVyTH, está inspirado
primordialmente en las siguientes fuentes:

\begin{enumerate}
\def\labelenumi{\arabic{enumi})}
\tightlist
\item
  La Encuesta de Viajes y Turismo en Hogares realizada conjuntamente
  por el Instituto Nacional de Estadística y Censos y la Secretaría de
  Turismo de la Nación durante el año 2006 (EVyTH-06).
\item
  Los trabajos realizados a fines de 2010 para la Prueba Piloto para
  la Encuesta de Turismo Doméstico en Hogares Residentes (Prueba
  Piloto).
\item
  La Encuesta de Turismo Doméstico en Hogares Residentes para medir el
  período estival de 2011 (la EVyTH-11).
\item
  La Encuesta de Turismo Doméstico en Hogares Residentes para medir el
  período de abril de 2011.
\item
  La Encuesta de Turismo Doméstico a Hogares Residentes para el
  periodo 2011/2012 (la EVyTH-12).
\item
  La Encuesta de Viajes y Turismo de los Hogares (la EVyTH-13).
\item
  La Encuesta de Viajes y Turismo de los Hogares (la EVyTH-14).
\item
  La Encuesta de Viajes y Turismo de los Hogares (la EVyTH-15).
\item
  La Encuesta de Viajes y Turismo de los Hogares (la EVyTH-16).
\item
  La Encuesta de Viajes y Turismo de los Hogares (la EVyTH-17).
\item
  La Encuesta de Viajes y Turismo de los Hogares (la EVyTH-18).
\item
  La Encuesta de Viajes y Turismo de los Hogares (la EVyTH-19).
\item
  La Encuesta de Viajes y Turismo de los Hogares (la EVyTH-20).
\end{enumerate}

\hypertarget{antecedentes-internacionales-de-la-evyth}{%
\subsubsection{\texorpdfstring{\textbf{ANTECEDENTES INTERNACIONALES DE LA EVYTH}}{ANTECEDENTES INTERNACIONALES DE LA EVYTH}}\label{antecedentes-internacionales-de-la-evyth}}

La relevancia de las encuestas continuas de turismo doméstico ha sido
recogida por la Organización Mundial del Turismo (OMT) cuando en el
párrafo 2.72 de las Recomendaciones Internacionales sobre Estadísticas
de Turismo 2008 se indica que ``Las encuestas a hogares basadas en
muestras estratificadas usando criterios espaciales, demográficos y
socio- económicos puede ser un instrumento eficiente y apropiado para
medir la actividad del turismo doméstico y su gasto relacionado''.

El órgano rector de la estadística de la Unión Económica Europea
(EUROSTAT), ha receptado estas recomendaciones exigiendo como Directiva
Comunitaria una encuesta de turismo interno para cada país miembro. De
esta forma, a modo de ejemplo, se pueden citar los casos de España (con
la ``Familitur'') y el Reino Unido (con la ``United Kingdom Tourism
Survey'').

A propósito de ello, se observa en el Cuadro 1, el resultado del
relevamiento realizado con el fin de identificar aquellos países que han
instrumentado encuestas de turismo similares a la EVyTH. Como se
observa, con color azul se han marcado aquellos países que miden el
Turismo Nacional a través de encuestas específicas de turismo, sin
perjuicio de que la frecuencia de las mediciones no necesariamente es
realizada con periodicidad mínima anual (es decir, en esta categoría
pueden encontrarse países que realizan solo esporádicamente encuestas
específicas de turismo). Entre estos países podemos citar a México,
China y Brasil\footnote{Por ejemplo, en Brasil existe una encuesta específica de
  turismo tendiente a medir el Turismo Nacional (la Estudio de Demanda
  Turística Nacional) cuya frecuencia no es, como mínimo, regularmente
  anual (en efecto, se cuentan con datos para los siguientes años:
  1998, 2002, 2006, 2007 y 2012).}. La segunda clasificación (en color celeste
claro), contiene aquellos países que miden el Turismo Nacional,
periódicamente, con una frecuencia mínima anual. Sin embargo, dentro de
esta clasificación pueden hallarse países que no han implementado una
encuesta específica para medir el turismo (por ejemplo, algunos países
dentro de esta clasificación miden el turismo a través de encuestas
permanentes a hogares). Algunos de los países que se encuentran en esta
categoría son Croacia, Paraguay y Uruguay.

La tercera clasificación, de color turquesa, es una intersección entre
las dos caracterizaciones antes mencionadas (es decir, entre el color
azul y color celeste claro).

Aquí se encuentran los países que miden el Turismo Nacional a través de
encuestas específicas de turismo y presentan, además, un relevamiento
periódico con una frecuencia como mínimo anual. Alguno de los países que
se encuentran dentro de este grupo son Argentina, Gran Bretaña, España y
Japón.~La categoría restante, agrupa los países que no presentan
(publican) mediciones sobre Turismo Nacional.

Como se observa, entonces, esta Encuesta de Viajes y Turismo de los Hogares encargada por el Organismo Contratante coloca a la Argentina entre un conjunto de países selectos que han implementado encuestas específicas para medir y caracterizar el Turismo Nacional con una periodicidad mínima anual. Es relevante destacar que este esfuerzo por parte del Organismo Contratante, permite contar con información crucial para diseñar y evaluar la implementación de políticas de desarrollo turístico, al tiempo que constituye una valiosa información para los distintos actores del sector.

\textbf{Cuadro 1: Países que relevan Encuestas de Turismo Doméstico}

\hypertarget{planeamiento-de-los-aspectos-metodoluxf3gicos-y-tuxe9cnicos-de-la-evyth}{%
\chapter{\texorpdfstring{\textbf{PLANEAMIENTO DE LOS ASPECTOS METODOLÓGICOS Y TÉCNICOS DE LA EVYTH}}{PLANEAMIENTO DE LOS ASPECTOS METODOLÓGICOS Y TÉCNICOS DE LA EVYTH}}\label{planeamiento-de-los-aspectos-metodoluxf3gicos-y-tuxe9cnicos-de-la-evyth}}

A continuación, se detallan los aspectos metodológicos sobre los cuales se estructurará el relevamiento de la EVyTH-21.
Por consiguiente, se brinda una descripción pormenorizada sobre la unidad de observación a considerar, el marco muestral a utilizar, el diseño muestral a adoptar, el proceso de calibración a implementar, etc.
Se pretende con este capítulo dar cumplimiento por El Pliego en términos de explicitar la metodología a utilizarse durante el relevamiento.
Además de prefijar el marco, creemos también importante el contenido de este capítulo con miras a asentar un precedente que sirva de legado para que futuros relevamientos puedan replicar la metodología y procedimientos involucrados en la toma de datos y su posterior procesamiento para así conservar la consistencia y consecuente comparabilidad de los resultados.

\hypertarget{unidades-de-observaciuxf3n}{%
\section{\texorpdfstring{\textbf{UNIDADES DE OBSERVACIÓN}}{UNIDADES DE OBSERVACIÓN}}\label{unidades-de-observaciuxf3n}}

Siguiendo las definiciones establecidas por El Pliego (las que son tomadas, a su vez, de la Prueba Piloto, la EVyTH-11, la EVyTH-12, la EVyTH-13, la EVyTH-14, la EVyTH-15, la EVyTH- 16, la EVyTH-17, la EVyTH-18, la EVyTH-19 y la EVyTH-20), las unidades de observación de la EVyTH son las personas que residen en hogares particulares dentro del territorio argentino.
Se considera hogar particular a aquel hogar constituido por una persona o grupo de personas, parientes o no, que conviven en una misma vivienda bajo un régimen de tipo familiar y consumen alimentos con cargo al mismo presupuesto.
Asimismo, se considera miembros del hogar a las personas que habitan en una misma vivienda bajo un régimen de tipo familiar, comparten sus gastos de alimentación, habitan la vivienda desde hace 6 o más meses o, si viven en ella hace menos de 6 meses, han fijado o piensan fijar allí su residencia.

\hypertarget{marco-conceptual-definiciuxf3n-de-viaje}{%
\section{\texorpdfstring{\textbf{MARCO CONCEPTUAL: DEFINICIÓN DE VIAJE}}{MARCO CONCEPTUAL: DEFINICIÓN DE VIAJE}}\label{marco-conceptual-definiciuxf3n-de-viaje}}

Siguiendo las definiciones habituales y las recomendaciones internacionales de la Organización Mundial del Turismo, a los efectos del relevamiento de la EVyTH, se define como viaje de turismo a todo aquel desplazamiento realizado por todos, algunos o al menos uno de los miembros del hogar fuera de su entorno habitual (las segundas viviendas por definición no forman parte).
Debe tenerse en cuenta que:

\begin{itemize}
\item
  Debe tener una duración inferior a un año y quien viajó no haber fijado su residencia en el lugar de destino.
\item
  Comprende los traslados, estancias y todas las actividades que las personas realizan en el lugar de destino.
\item
  Puede haber ido a un sólo destino o haber tenido varias etapas.
\end{itemize}

Los motivos del viaje pueden ser esparcimiento, ocio o recreación; visita a familiares y amigos; negocios o motivos profesionales; estudios y formación; razones de salud, religión u otros (trámites, compras, etc.).

Se incluyen tanto los viajes en los que el viajero pernocta al menos una noche en el lugar visitado como las visitas de un día (excusiones sin pernocte).

Se considera viaje a todos los desplazamientos que cumplan con las condiciones enumeradas anteriormente y en los que hayan participado todos, algunos o uno de los miembros actuales del hogar encuestado.

El entorno habitual comprende la zona cercana al lugar de residencia habitual del hogar más todos los lugares visitados frecuentemente, aunque no sean cercanos a su lugar de residencia.
Operacionalmente, la distancia que abarca el entorno habitual se define como el radio de 20 Km.
desde la ciudad/localidad en que se encuentra la vivienda principal del hogar, a excepción del Gran Buenos Aires (la Ciudad Autónoma de Buenos Aires más los 24 Partidos que conforman el Conurbano de esta provincia), donde la distancia que determina el entorno habitual es el radio de 40 Km.
de la localidad donde reside habitualmente el hogar.
La frecuencia mínima de visita a un sitio, para que éste sea considerado como parte del entorno habitual, es de una vez por semana.
Por lo tanto, los viajes que aquí consideramos son los realizados a destinos ubicados a más de 20/40 Km.
de la localidad donde está la vivienda principal y que fueron visitados con una frecuencia no semanal.

En el caso de los desplazamientos internacionales, el criterio de distancia mínima es reemplazado por el de traspaso de la frontera administrativa.
No obstante, el criterio de frecuencia sigue rigiendo, por lo cual, si una persona habitualmente traspasa la frontera una o más veces por semana, se considera que estos desplazamientos son parte de su entorno habitual.

Un desplazamiento a un lugar fuera del entorno habitual, en términos de distancia y frecuencia, puede no ser un viaje.
Estas exclusiones rigen en todos los casos, incluso cuando el viaje se realice a una segunda vivienda del hogar\footnote{Si una persona realizó un desplazamiento cuyo motivo principal fue trabajar como empleado de una empresa o institución residente en el lugar visitado (a cambio de un salario, se trate de un empleo en ``blanco'' o no) no se considerará visitante de ese lugar y, por tanto, no se registrará el viaje.
  No se considera ``empleado'' si quien viaja por trabajo brinda un servicio por el cual percibe honorarios, o bien si recibe dinero en concepto de ``viáticos'', es decir, una asignación para cubrir los gastos del viaje (pasajes, alojamiento, alimentación, etc.).
  Por ejemplo, no se consideran viajeros cuando son empleados en el lugar de destino:

  \begin{itemize}
  \item
    trabajadores en temporada en hoteles, explotaciones agropecuarias, etc.;
  \item
    trabajadores temporarios en internados, hospitales y similares;
  \item
    estudiantes que viajan por períodos inferiores a 12 meses y trabajan para mantenerse (tanto quienes reciben un salario por su trabajo como quienes ayudan a una familia en las labores de la casa y el cuidado de los niños a cambio de la estancia y el mantenimiento);
  \item
    quienes viajan como acompañantes (familiares, servicio doméstico, etc.) de aquellos que lo hacen por cualquiera de los motivos ya enunciados, aunque hayan realizado actividades turísticas.
  \end{itemize}

\begin{Verbatim}
En cambio, si se consideran que realizaron viajes aquellas personas que viajaron por trabajo, pero no fueron empleados en el
lugar de destino, por ejemplo:
\end{Verbatim}

  \begin{itemize}
  \item
    conferenciantes, artistas de espectáculos y consultores;\\
  \item
    un abogado que defiende en un juicio a una persona o empresa del lugar visitado;
  \item
    un cirujano que viaja a realizar una operación por única vez.

    No se consideran viajes aquellos desplazamientos realizados por las personas para quienes el traslado es parte de su trabajo.

    Por ejemplo, los pilotos de avión, conductores de camión o de ómnibus de larga distancia, viajantes de comercio y similares (incluyendo personal de seguridad, de asistencia -- azafatas-, etc.).
  \end{itemize}}.

El Cuadro 2 a continuación resume la definición de entorno habitual discutida anteriormente.

\textbf{Cuadro 2: Definición de Entorno Habitual}

\hypertarget{unidades-de-anuxe1lisis}{%
\section{\texorpdfstring{\textbf{UNIDADES DE ANÁLISIS}}{UNIDADES DE ANÁLISIS}}\label{unidades-de-anuxe1lisis}}

El marco conceptual antes definido permitirá obtener conclusiones a partir de diversas unidades de análisis.
A continuación, se las repasan junto con una breve descripción de cada una de ellas:

\begin{itemize}
\item
  \emph{Hogar}: es la persona o grupo de personas, parientes o no, que habitan bajo un mismo techo en un régimen de tipo familiar; es decir, comparten sus gastos en alimentación con cargo a un mismo presupuesto.
\item
  \emph{Persona}: cada uno de los individuos integrantes de un hogar.
\item
  \emph{Viaje}: significa cada desplazamiento que realiza una persona fuera del entorno habitual pernoctando en el/los lugar/es de destino.
\item
  \emph{Visita de un día}: es el recorrido de una persona fuera del entorno habitual sin pernoctar en el lugar de destino.
\item
  \emph{Pernoctación}: se trata de una noche de alojamiento de una persona.
\end{itemize}

Ver en la sección 6.5.1 (el Manual del Encuestador) para más detalles de cada concepto.

\hypertarget{universo-bajo-estudio}{%
\section{\texorpdfstring{\textbf{UNIVERSO BAJO ESTUDIO}}{UNIVERSO BAJO ESTUDIO}}\label{universo-bajo-estudio}}

Según lo estipulado por El Pliego, el universo bajo estudio de la encuesta serán los aglomerados urbanos que tengan más de 100.000 habitantes.
En aquellas provincias donde no haya ningún aglomerado que cumpla dicha condición, se incorporará el aglomerado compuesto por la capital de la misma\footnote{Se sigue aquí la metodología utilizada durante la EVyTH-06.}.
Por lo tanto, se considerarán 32 aglomerados según la descripción que se anexa en el Cuadro 22 del Anexo.
Los aglomerados se agruparán en las siguientes Regiones\footnote{Las regiones están definidas a partir del criterio que la Secretaría de Turismo (y luego el Organismo Contratante) ha venido utilizando para sus relevamientos (EVyTH-06, Encuesta de Ocupación Hotelera, Plan Federal Estratégico de Turismo Sustentable, etc.).},
\footnote{Se debió decidir en qué regiones incluir los aglomerados urbanos ``San Nicolás -- Villa Constitución'' y ``Viedma -- Carmen de Patagones'' puesto que las ciudades se encuentran en distintas provincias y, por ende, en diferentes regiones (según la definición que sigue).
  Se incluye al primer aglomerado (San Nicolás -- Villa Constitución) en la Región Buenos Aires, mientras que el segundo (Viedma -- Carmen de Patagones), en la Región Patagonia.}:

\begin{itemize}
\item
  Región Ciudad de Buenos Aires.
\item
  Partidos del Conurbano de la Provincia de Buenos Aires.
\item
  Región Interior de la Provincia de Buenos Aires: compuesto por todos los aglomerados pertenecientes a dicha Provincia, excepto Partidos del Conurbano de la Provincia de Buenos Aires.
\item
  Región Córdoba: compuesto por todos los aglomerados pertenecientes a la Provincia de Córdoba.
\item
  Región Litoral: compuesto por todos los aglomerados pertenecientes a las Provincias de Santa Fe, Entre Ríos, Corrientes, Misiones, Formosa y Chaco.
\item
  Región Norte: compuesto por todos los aglomerados pertenecientes a las provincias de Jujuy, Salta, Tucumán, Santiago del Estero, Catamarca y La Rioja.
\item
  Región Cuyo: compuesto por todos los aglomerados pertenecientes a las provincias de Mendoza, San Luis, San Juan.
\item
  Región Patagónica: compuesto por todos los aglomerados pertenecientes a las provincias de La Pampa, Neuquén, Río Negro, Chubut, Santa Cruz y Tierra del Fuego\footnote{Cabe aclarar que se incluirán el aglomerado ``San Nicolás -- Villa Constitución'' en la Región Provincia de Buenos Aires, mientras que el aglomerado ``Viedma -- Carmen de Patagones'' será incluido en la Región Patagonia.}.
\end{itemize}

A continuación, se presenta un cuadro que, a modo ilustrativo, indica los aglomerados que serán relevados.
Para mayor detalle véase el Anexo.

\textbf{Cuadro 3: Mapa con los aglomerados capitales de provincia o aglomerados urbanos de más de 100.000 habitantes del universo bajo estudio}

\hypertarget{diseuxf1o-muestral}{%
\section{\texorpdfstring{\textbf{DISEÑO MUESTRAL}}{DISEÑO MUESTRAL}}\label{diseuxf1o-muestral}}

El diseño muestral a considerar durante el relevamiento de la EVyTH se estructura en las siguientes pautas:

\begin{enumerate}
\def\labelenumi{\arabic{enumi})}
\tightlist
\item
  Cada una de las regiones definidas es considerará como estrato y se seleccionará muestras independientes en cada uno de ellos.
\item
  El método de selección a implementar es el sistemático.
\item
  Si fuera imposible disponer de información auxiliar que permita realizar estratificaciones socioeconómicas en cada una de las regiones, previo a la selección, se procederá a ordenar las bases de manera tal de salvar este inconveniente con el objetivo de lograr ``una estratificación implícita''.
\item
  Cada región, entonces, se ordenará por Aglomerado -- Partido/Departamento -- Localidad -- Característica Telefónica.
\end{enumerate}

\hypertarget{tamauxf1o-muestraly-manejo-de-los-reemplazos}{%
\section{\texorpdfstring{\textbf{TAMAÑO MUESTRALY MANEJO DE LOS REEMPLAZOS}}{TAMAÑO MUESTRALY MANEJO DE LOS REEMPLAZOS}}\label{tamauxf1o-muestraly-manejo-de-los-reemplazos}}

La Prueba Piloto de la EVyTH, contribuyó significativamente a probar numerosas cuestiones vinculadas al proceso de encuestamiento, a saber: tecnología a utilizar, dimensionamiento de equipos, delineamiento del cuestionario, etc.
Tuvo, también, un muy valioso aporte desde el punto de vista del muestreo.
En efecto, la Prueba Piloto estuvo centrada en probar si el padrón telefónico con el que se contó constituye un buen marco de muestreo para extraer la muestra definitiva.
Luego, la EVyTH-11 implementó la metodología especificada a partir de la Prueba Piloto y se concluyó que necesitan entre cinco y seis números telefónicos por hogar para obtener una encuesta efectiva.

Consecuentemente, en base a estos resultados y en cumplimiento con lo establecido por El Pliego\footnote{Véase Pliego página 28}, la EVyTH-21 considerará una muestra titular de 2.600 hogares distribuidos a lo largo de cada Región - Aglomerado -- Partido/Departamento - Localidad (ver Cuadro 3).
Asimismo, en caso de que la muestra titular no pueda ser completada (por ausencia de respondentes, rechazo a responder la encuesta o errores del marco), se seleccionarán, con el mismo diseño muestral, seis hogares adicionales como reemplazos por hogar titular\footnote{La posibilidad de disponer de la Guía Telefónica Nacional actualizada reduce considerablemente este problema ya que, por ejemplo, elimina los teléfonos deshabilitados y, de ese modo, el relevamiento resulta más eficiente.}
. Los reemplazos deberán coincidir en la Región, Aglomerado, Partido/Departamento y Localidad y serán utilizados luego de, al menos, seis intentos de comunicación insatisfactorios con la muestra inicial
. En conclusión, se tomarán 6 reemplazos y, por lo tanto, la muestra tendrá un tamaño de 15.000 hogares con sus respectivos números de teléfono (por onda)
.

De esta forma, con el mecanismo de períodos de referencia y ventanas de observación que se implementará (ver más adelante sección 2.6.2), se logrará que, para un mes determinado, se duplique el tamaño de muestra (es decir, se relevarán 5.200 hogares)\footnote{Esta metodología se inspira en la Familitur de España y sigue lo pautado por El Pliego.}.
La muestra completa (incluyendo reemplazos) estará en constante disposición del Departamento de Estadísticas del Organismo Contratante bajo las reservas del secreto estadístico contemplado en El Pliego.

Finalmente, en base a resultados de las experiencias anteriores de la Encuesta, y en cumplimiento con lo establecido por El Pliego, las expectativas de viajes para los periodos de alta demanda de turismo como el invierno y el verano considerarán una muestra titular de 1.500 hogares distribuidos a lo largo de cada Región - Aglomerado -- Partido/Departamento - Localidad.

\hypertarget{marco-de-muestreo}{%
\subsection{\texorpdfstring{\textbf{MARCO DE MUESTREO}}{MARCO DE MUESTREO}}\label{marco-de-muestreo}}

La EVyTH se instrumentará a través de un sistema C.A.T.I. (Computer Assisted Telephone Interviewing).
Para ello, se utilizará como marco de muestreo la Guía Telefónica Nacional (GTN) del año 2011 (actualizada a 2012/13/14/15/16/17/18/19/20).
MacroConsulting cuenta con una Guía ya consistida y validada.

Los Licenciatarios de Telefonía Básica es la fuente consultada para obtener la GTN.
La GTN está compuesta por los siguientes campos:

\begin{itemize}
\item
  Rubro/Actividad
\item
  Apellido y Nombre
\item
  Domicilio
\item
  Teléfono
\item
  CP
\item
  Ciudad
\item
  Localidad
\item
  Provincia
\end{itemize}

Con el fin de considerar únicamente aquellos registros que corresponden a los hogares (y no tener en cuenta comercios, empresas, entidades gubernamentales, etc.), se realizaron ciertos filtros a dicha base de datos.

\begin{itemize}
\item
  En primer lugar, se eliminaron todos aquellos datos cuyo campo ``Rubro/Actividad'' no fuera ``Part/Com''.
  Luego se procedió a eliminar los registros en los que o bien la primera o bien la última palabra del campo ``Apellido y Nombre'' hicieran referencia a empresas, comercios, entes gubernamentales, etc.
  Por ejemplo, se descartaron aquellos datos cuya última palabra fuera ``S.A.'', ``S.R.L.'', etc., dado que hacen referencia a sociedades.
  Otros ejemplos son los registros cuya primera palabra del campo ``Apellido y Nombre'' fuera ``Hospital'', ``Gobierno'', etc.
  De esta forma se eliminaron gran parte de los datos que no correspondieran a hogares.
\item
  Se eliminaron los datos repetidos, considerándose como tales aquellos con un mismo nombre y dirección.
\item
  Finalmente, en el caso de la Ciudad Autónoma de Buenos Aires, se utilizó una base de datos de códigos postales mediante la cual se agregó a cada registro de la base de la GTN su barrio correspondiente.
\item
  En base a esta metodología se actualizó el Marco de Muestreo utilizado para la Prueba Piloto, la EVyTH-11, la EVyTH-12, EVyTH-13 EVyTH-14, EVyTH-15, EVyTH-16 EVyTH-17, EVyTH-18, EVyTH-19 y la EVyTH-20 para considerar los cambios y renovaciones que podrían haber ocurrido desde mediados de 2011 en adelante.
\end{itemize}

\hypertarget{periodicidad-ventana-de-observaciuxf3n-peruxedodos-de-referencia-y-esquema-de-rotaciuxf3n-muestral}{%
\subsection{\texorpdfstring{\textbf{PERIODICIDAD, VENTANA DE OBSERVACIÓN, PERÍODOS DE REFERENCIA Y ESQUEMA DE ROTACIÓN MUESTRAL}}{PERIODICIDAD, VENTANA DE OBSERVACIÓN, PERÍODOS DE REFERENCIA Y ESQUEMA DE ROTACIÓN MUESTRAL}}\label{periodicidad-ventana-de-observaciuxf3n-peruxedodos-de-referencia-y-esquema-de-rotaciuxf3n-muestral}}

\hypertarget{periodicidad-ventana-de-observaciuxf3n-y-peruxedodos-de-referencia}{%
\subsubsection{\texorpdfstring{\textbf{PERIODICIDAD, VENTANA DE OBSERVACIÓN Y PERÍODOS DE REFERENCIA}}{PERIODICIDAD, VENTANA DE OBSERVACIÓN Y PERÍODOS DE REFERENCIA}}\label{periodicidad-ventana-de-observaciuxf3n-y-peruxedodos-de-referencia}}

Para llevar adelante el campo de la EVyTH, siguiendo lo pautado por El Pliego, se propone utilizar una periodicidad de muestreo mensual; es decir, todos los meses se releva una muestra seleccionada de acuerdo a la metodología descripta anteriormente.

Además, se contemplará una ventana de observación\footnote{El período para el cual se brinda información se denomina ventana de observación.} bimestral para los meses a ser relevados, es decir, períodos de recordación de dos meses consecutivos.
Las ventanas de observación de períodos cortos, como este caso, eliminan la necesidad de recordación por tiempos prolongados, lo que favorece la veracidad de la información.

Tomar períodos bimestrales como ventanas de observación permite generar un sistema de solapamiento que brinda mayor robustez a las conclusiones que se alcancen.

Para dar mayor detalle del mecanismo a implementar en términos de ventanas de observación y solapamiento, se transcribe en el Cuadro 4, la conceptualización del esquema.

\textbf{Cuadro 4: Periodicidad, Ventana de Observación y Períodos de Referencia}

De acuerdo con El Pliego, la EVyTH debe desarrollarse ``durante doce ondas consecutivas, a razón de una onda por mes calendario''\footnote{Véase Pliego pág.
  25.}.
El mes calendario de comienzo del relevamiento es marzo de 2021.
Por lo tanto, para ese mes (que sería la Onda 1 de la EVyTH-21), se encuestará la muestra (titular) de 2.600 hogares (con los remplazos que hicieran falta) y se consultará por los períodos de referencia correspondiente a los dos meses calendarios anteriores; es decir, enero y febrero de 2021.
Durante el segundo mes de relevamiento, abril de 2021 (Onda 2), se procederá de forma similar para los viajes realizados en los dos meses anteriores (febrero y marzo de 2021).
Consecuentemente el Mes 2 (abril de 2021) quedará ``solapada'' la muestra, habiéndose encuestado 5.200 hogares para este mes (es decir, luego del segundo mes de relevamiento).
De este modo se garantizará que, para las estimaciones contenidas en los informes correspondiente a cada mes, en el apartado de Anticipo Mensual (véase sección 3.1.1.1.1) serán obtenidas a partir de 2.600 encuestas efectivas, mientras que para las estimaciones mensuales finales la muestra quedará duplicada a 5.200 (lo que garantizará la robustez del resultado).
\footnote{Siguiendo lo especificado en El Pliego (ver página 29).}

\hypertarget{esquema-de-rotaciuxf3n-muestral}{%
\subsubsection{\texorpdfstring{\textbf{ESQUEMA DE ROTACIÓN MUESTRAL}}{ESQUEMA DE ROTACIÓN MUESTRAL}}\label{esquema-de-rotaciuxf3n-muestral}}

Con el objetivo de mejorar la comparación interanual de las principales variables a relevar, se renovará periódicamente el conjunto de hogares a encuestar (panel de respondentes)\footnote{La forma en que se produce esta renovación se denomina esquema de rotación.}.
Conceptualmente, el esquema de rotación de una encuesta tiene incidencia sobre los siguientes aspectos:

\begin{itemize}
\item
  Precisión de las estimaciones del cambio entre dos períodos diferentes.
\item
  Precisión de las estimaciones obtenidas al agregar muestra.
\item
  Nivel de no respuesta (por cansancio del panel).
\end{itemize}

El solapamiento de las muestras entre dos períodos consecutivos (o sea el porcentaje de muestra en común) juega en sentido contrario para los primeros dos aspectos: si un esquema tiene un alto porcentaje de solapamiento entre un período y el siguiente, puede medir adecuadamente los cambios, pero disminuir su precisión para una agregación a lo largo de varios períodos.
Por el contrario, un bajo solapamiento mejora la precisión cuando se agrega muestra, pero disminuye la precisión de la estimación del cambio entre períodos sucesivos.

El objetivo primordial de la EVyTH es medir la evolución de los viajes que, es sabido, es una variable que presenta una gran estacionalidad.
Consecuentemente, la medición de la evolución no se realizará a partir de comparar dos meses consecutivos (excepto algunos meses en particular), sino que anualmente o agregando períodos (estival, invernal, etc.).

Tal como se describió anteriormente, de acuerdo a El Pliego, el manejo muestral exige escoger la muestra antes de cada mes de relevamiento de modo tal de obtener muestras independientes, pero solapadas.
El esquema de rotación que se implementará consistirá en incluir en la muestra del mes 1 del año t, la muestra titular de los hogares respondentes del mismo periodo (mes 1) del año anterior (t-1), no con el objetivo de estudiar el comportamiento de esos hogares en particular, sino de modo tal de hacer que las variaciones interanuales no presenten diferencias extremas\footnote{Este esquema ya fue propuesto como parte de La Oferta de MacroConsulting.}.
De esta forma se maximiza la precisión en cuanto a los cambios suscitados entre un año y otro.
Un sistema de rotación muestral como este, permite evitar el cansancio del panel respondente ya que se repetirán los hogares luego de doce meses.
De esta manera, se logrará una mejor ponderación entre ganancia en precisión (al rotar la muestra) y medición del cambio entre dos períodos (dos años).

Es relevante señalar que durante la EVyTH-12 se llevó a cabo una prueba piloto de esta metodología ``de panel'' entre los meses de abril y mayo (ondas 3 y 4).
Los resultados fueron positivos ya que no se encontraron sesgos en las características de los respondentes\footnote{El control consistió en evaluar las características principales de los respondentes de la encuesta durante las primeras cuatro ondas de relevamiento del año 2011 y del año 2012.
  El objetivo fue estudiar si se producía algún sesgo demográfico al utilizar muestras panel.
  Los relevamientos de las ondas 3 y 4 (relevamientos de abril y mayo) de la EVyTH-12 fueron panel (incluyeron a los hogares respondentes del mismo periodo del año anterior), mientras que las ondas 1 y 2 (febrero y marzo) no lo fueron.

  Los resultados del estudio demostraron que no se encontraron variaciones significativas en ningunas de las variables estudiadas (edad promedio edad en tramos, relación de parentesco del miembro respondente con el jefe de hogar, asistencia a establecimiento educativo, máximo nivel educativo alcanzado, situación ocupacional) entre los meses con muestra panel y los meses sin muestra panel.}, a la vez que se aumentó la tasa de efectividad de las encuestas\footnote{Se estudiaron las características de los respondentes no notándose diferencias estadísticamente significativas entre los respondentes que aceptaron ser encuestados el año anterior respecto del mismo grupo que aceptó ser encuestado en el 2012.}
. En efecto, en comparación con las ondas en las que no fue implementada esta metodología de panel (es decir, para las ondas 1 a 4 de la EVyTH-11 y ondas 1 y 2 de la EVyTH-12), la tasa de efectividad de encuesta para los hogares titulares durante las ondas en las que sí se adoptó, (es decir, ondas 3 y 4) fue del doble
. Mientras que se halló que poco más de dos de cada diez hogares de la muestra titular decidieron responder la encuesta en las primeras ondas, casi cinco de cada diez decidieron hacerlo en las ondas 3 y 4 durante el relevamiento de la EVyTH-12
. Además, se realizó un control por características sociodemográficas del miembro respondente para verificar que no se hubiera sesgado la muestra
. Las variables consideradas fueron
:

\begin{itemize}
\item
  edad promedio del miembro respondente
\item
  jefe de hogar o relación de parentesco con el jefe de hogar
\item
  asistencia a establecimiento educativo
\item
  máximo nivel educativo alcanzado
\item
  situación ocupacional
\end{itemize}

El estudio arrojó el resultado antes comentado: no se encontraron mayores diferencias socio-económicas entre aquellos que accedieron a contestar las preguntas de esta edición de la encuesta con aquellos que lo hicieron en las ondas con distinta metodología.

\hypertarget{cuestionario}{%
\section{\texorpdfstring{\textbf{CUESTIONARIO}}{CUESTIONARIO}}\label{cuestionario}}

Siguiendo lo especificado por El Pliego\footnote{El Pliego especifica: "Se anexa formulario base para el relevamiento de la encuesta mensual y el de expectativas de viajes.

  Los cambios en los formularios base, que no se deban a cambios en los períodos de referencia de las distintas ondas, se realizarán únicamente a solicitud del personal técnico de la Secretaría, aunque en la propuesta se podrán realizar sugerencias de modificación del formulario que serán consideradas al evaluar las ofertas" (ver hoja 30).} a continuación se expone el Cuestionario que se implementará durante el relevamiento de la EVyTH.
El mismo consiste en un Cuestionario Base (o sección ``pétrea'') que, básicamente, pretende relevar los viajes realizados en los últimos dos meses y un Cuestionario ``ad-hoc'' que tiene la función de captar eventuales viajes futuros para fechas especiales como son los periodos de Verano/Invierno.

\hypertarget{antecedente-de-la-actual-edicion-de-la-evyth}{%
\subsection{\texorpdfstring{\textbf{ANTECEDENTE DE LA ACTUAL EDICIÓN DE LA EVYTH}}{ANTECEDENTE DE LA ACTUAL EDICIÓN DE LA EVYTH}}\label{antecedente-de-la-actual-edicion-de-la-evyth}}

El Cuestionario implementado en la EVyTH-06, constituyó el modelo sobre el cual se trabajó para llegar al cuestionario aplicado en la Prueba Piloto, en la EVyTH-11, en la EVyTH-12, en la EVyTH-13, en la EVyTH-14, en la EVyTH-15, EVyTH-16, en la EVyTH-17, en la EVyTH-18, en la EVyTH-19 y en la EVyTH-20 que son los antecedentes a la actual edición de la EVyTH.

Al observarse el Formulario anexado a El Pliego, se puede notar que la estructura conceptual del cuestionario ha perdurado desde su comienzo en 2010 aunque, sin embargo, seguramente fue preciso realizar algunas modificaciones atendiendo a situaciones de diversa índole\footnote{Circunstancias a las que se le debe adicionar el hecho de que ahora el relevamiento es telefónico (a diferencia de la EVyTH-06, que se realizó cara a cara).}.
Los cambios que ha sufrido el cuestionario se resumen a continuación y resultan en el Formulario Base anexado a este informe (ver sección 6.6).
\footnote{Este Formulario Base ya incorpora la actualización de los tramos de gastos (según tipo de viaje) y de ingreso (así como también los cambios necesarios a las preguntas de comportamiento turístico).}

En primer lugar, al comenzar la edición 2012 de la EVyTH se incorporó un relevamiento ``ad- hoc'' (complementario al cuestionario básico) con el fin de relevar las expectativas de viaje de los hogares argentinos durante fechas especiales dentro de cada mes de relevamiento.

Para esta edición se anexa el formulario de expectativas para viajes de las temporadas de invierno y verano (ver sección 6.7).

En segundo lugar, durante la EVyTH-12, se incorporó al cuestionario base implementado durante la EVyTH-11, la categoría ``educación especial sin nivel'' a la pregunta de nivel educativo.
Esto permitió categorizar con mayor precisión a todos los encuestados.

En tercer lugar, se modificó la manera de formular algunas preguntas para hacerlas más comprensibles.
Por ejemplo, en la pregunta sobre duración del viaje de corta duración a segundas viviendas se remplazó ``¿Cuántas noches pasaron en la vivienda?'' por ``¿Cuál fue la duración del viaje, desde que salieron hasta que regresaron a su hogar?'' .

En cuarto lugar, se reemplazó en la pregunta ``¿Qué medio de transporte utilizaron para ir hasta allá? La categoría''Automóvil o similar (propio)" por solo ``automóvil'', por prestarse a confusión.

En quinto lugar, en el bloque donde se indaga sobre las características de los miembros del hogar, se realizó una modificación respecto a las características ocupacionales del miembro del hogar.
Se cambió la categoría ``jubilado/a'' por ``jubilado/a o pensionado/a''.

En sexto lugar, se actualizaron los tramos de gasto e ingresos del hogar.
Debido a la evolución en el ingreso de los hogares y del consecuente gasto turístico, resultó necesario realizar un ajuste en los montos de tramos de gasto de total del viaje y del ingreso del hogar.
Además, se cuenta con información acumulada que permite realizar un análisis exhaustivo al interior de la propia fuente con el fin de optimizar las escalas de los tramos.

Otra modificación realizada al cuestionario base (o sección ``pétrea'') fue la re estructuración de las preguntas sobre las razones que llevaron al hogar a elegir el destino al que finalmente viajaron y si ya habían visitado ese lugar con anterioridad.
Ambas preguntas fueron incorporadas en julio de 2012 con el fin de obtener información más específica y certera sobre los fundamentos que impulsaron al hogar a optar por ese destino y no por cualquier otro.
Así, los hogares deben definir si los atractivos naturales del lugar, la oferta histórica y cultural, la cercanía, los precios, la sugerencia de familiares o amigos que fueron allí antes, la residencia de familiares o amigos o la publicidad audiovisual influyeron en la decisión del destino.
Además, se indaga sobre la eventual evaluación de otro destino y sobre la visita previa de alguno de los miembros del hogar a ese destino.

Finalmente, se reformuló la pregunta sobre Comportamiento Turístico: para aquellos que viajaron en el año anterior, se indaga si viajaron dentro de Argentina y/o al Exterior (en reemplazo de la pregunta sobre si habían viajado por ocio o para visitar familiares -y la razón motivo) dado que esto no aportaba información valiosa.

\hypertarget{descripciuxf3n-del-formulario}{%
\subsection{\texorpdfstring{\textbf{DESCRIPCIÓN DEL FORMULARIO}}{DESCRIPCIÓN DEL FORMULARIO}}\label{descripciuxf3n-del-formulario}}

En la presente sección, se describe el cuestionario (anexado a El Pliego), que se utilizará para la presente edición de la EVyTH.
Este puede ser dividido en dos grandes secciones: un bloque ``pétreo'' (cuyo contenido solo esporádicamente sufre alguna alteración menor) y otro bloque conformado por preguntas que cambian en función de relevamientos ``ad-hoc'' que resultan de interés según el período y la ocasión.
Si bien esta sección es utilizada para indagar sobre la expectativa que los encuestados tienen en relación a la posibilidad de realizar algún viaje en un momento futuro (como los períodos de alta estacionalidad turística), puede ser utilizada, también, para llevar adelante encuestas de opinión u de otra índole\footnote{Su característica básica es que es completamente flexible permitiendo realizar relevamientos ``ad-hoc'' de forma fácil.} (durante la EVyTH-20 se indago sobre los perjuicios que tuvo la pandemia a causa del virus COVID-19 y sobre futuros viajes pos pandemia).

\hypertarget{cuestionario-base-secciuxf3n-puxe9trea}{%
\subsubsection{\texorpdfstring{\textbf{CUESTIONARIO BASE (SECCIÓN ``PÉTREA'')}}{CUESTIONARIO BASE (SECCIÓN ``PÉTREA'')}}\label{cuestionario-base-secciuxf3n-puxe9trea}}

Como se indicó anteriormente, la sección del cuestionario que se describe a continuación prácticamente no ha sufrido alteraciones desde el comienzo de la primera edición de la EVyTH en el año 2010\footnote{Ver sección anterior donde se describen las modificaciones introducidas a esta sección.} (para su consulta ver sección 6.6).
La idea es minimizar los cambios de contenido a esta parte del cuestionario con el fin de lograr la mayor compatibilidad entre los datos relevados en distintos momentos del tiempo y, de esta forma, generar una serie de tiempo lo más extensa posible.
Esta sección ``pétrea'' del cuestionario, se compone de 9 Bloques; a saber:

\begin{enumerate}
\def\labelenumi{\arabic{enumi}.}
\item
  Bloque ``Característicasdelosmiembrosdelhogar''.Estebloqueestásubdivido en tres.

  \begin{enumerate}
  \def\labelenumii{(\roman{enumii})}
  \tightlist
  \item
    Al inicio de la encuesta, sólo se indaga por la cantidad de personas que integran el hogar y las iniciales (de forma tal de mantener el anonimato) de cada uno de los miembros del hogar.
  \item
    Al final del cuestionario, luego de indagar por la realización de viajes y visitas y sus características, se indaga por las características de los miembros del hogar, bajo el supuesto de que, con el devenir de la encuesta, el entrevistado gana confianza y se muestra menos renuente a responder estas preguntas. En este mismo sub-bloque se indaga por el comportamiento turístico en el año calendario anterior a la encuesta (año 2020)\footnote{Este aspecto se indaga durante los meses de febrero hasta mayo inclusive.}; (iii) La indagación por los ingresos totales del hogar, que cierra la encuesta, se incluye como otro sub-bloque también al finalizar el Formulario Base.
  \end{enumerate}
\item
  Bloque ``Segundas Viviendas''.
  Se indaga únicamente por las segundas viviendas que el hogar dispuso y utilizó durante el periodo de referencia.
\item
  Bloque ``Viajes de corta duración a segundas viviendas'': 1 a 3 noches (VCDS).
\item
  Bloque``Viajesdelargaduraciónasegundasviviendas'':4nochesomás(LDSV).
\item
  Bloque``Visitasdeundíaasegundasviviendas''(VDSV).
\item
  Bloque``ViajesnoreiteradosenArgentina''(VA).
\item
  Bloque``Viajesnoreiteradosalexterior''(VE).
\item
  Bloque ``Viajes a destinos reiterados'' (DR).
  Este bloque funciona como ``reserva'', es decir, no se pregunta directamente al entrevistado si realizaron viajes de este tipo, sino que el encuestador lo completa en caso de detectar la existencia de este tipo de viajes en la indagación de los bloques anteriores (VA y VE).
\item
  Bloque``Visitasdeundía''.
\end{enumerate}

10.Bloque sobre viaje en períodos turísticos especiales.
Este relevamiento complementario indaga por la posibilidad de que alguno de los miembros del hogar haya realizado viajes en la/s fecha/s especiales durante los meses de referencia (por ejemplo, períodos de temporada de verano/invierno).
Es un bloque del cuestionario base que se ``activa'' automáticamente cuando, en los meses de referencia, ha habido un fin de semana largo.
En función de ello, se consulta explícitamente si el encuestado ha viajado durante ese momento específico del mes.

11.Comportamiento turístico de las personas.
Finalmente, el formulario también incorpora, solo durante los meses de febrero a mayo, una sección donde se indaga por viajes realizados durante al año anterior a modo de conocer el comportamiento turístico de las personas.
Con la información recabada en esta sección se confecciona el informe sobre Comportamiento Turístico de las Personas (ver sección 3 más adelante).

Sinópticamente, la lógica del Formulario Base puede resumirse a través del Cuadro 5:

\textbf{Cuadro 5: Esquematización del Cuestionario utilizado durante la EVyTH}

El cuestionario base contiene su propia terminología para caracterizar a los viajes (en sentido amplio).
Así, por ejemplo, cuando el visitante pernocta al menos una noche en el lugar de destino, se especifica como ``viajes'' y ``viajeros'' propiamente dichos.
Cuando la persona no pernocta en el lugar de destino, se hace referencia a ``visitas de un día'' y ``visitantes'' o ``excursionistas''.
Por lo tanto, en función de una serie de variables, se han definido diferentes tipos de viajes.
Estas variables son:

\begin{enumerate}
\def\labelenumi{\arabic{enumi}.}
\tightlist
\item
  Cantidad de noches en el lugar visitado. Se consideran las noches pasadas en alojamientos del lugar visitado.
\item
  Tipos de Alojamiento donde pernoctaron. Se clasifican en dos grupos:
\end{enumerate}

\begin{enumerate}
\def\labelenumi{\alph{enumi}.}
\tightlist
\item
  ``Segundas viviendas''
\item
  ``Otros tipos de alojamientos'': aquí se incluyen todos los tipos de alojamiento diferentes a las segundas viviendas (hoteles de todas las categorías, camping, casas de familiares o amigos, viviendas alquiladas por temporadas, etc.).
\end{enumerate}

\begin{enumerate}
\def\labelenumi{\arabic{enumi}.}
\setcounter{enumi}{2}
\tightlist
\item
  UbicacióndelDestinoPrincipal,segúnsea:
\end{enumerate}

\begin{enumerate}
\def\labelenumi{\alph{enumi}.}
\tightlist
\item
  En Argentina (ciudad/localidad).
\item
  En otro país.
\end{enumerate}

\begin{enumerate}
\def\labelenumi{\arabic{enumi}.}
\setcounter{enumi}{3}
\tightlist
\item
  Viajes Similares en los últimos 2 meses. De acuerdo a la cantidad de viajes realizados a un mismo destino con características similares (miembros participantes, cantidad de noches, tipo de alojamiento, motivo principal, etc.) los viajes se clasifican en:
\end{enumerate}

\begin{enumerate}
\def\labelenumi{\alph{enumi}.}
\tightlist
\item
  ``Reiterados'': cuando hayan realizado 3 o más viajes en los últimos 2 meses.
\item
  ``No reiterados'': cuando se hayan realizado uno o dos viajes en los últimos 2 meses.
\end{enumerate}

La información que se solicita depende del tipo de viaje que se trate y se registra en un Bloque específico del Formulario Base.
Esta clasificación responde a la estrategia definida para mejorar la indagación y captación de los distintos viajes que las personas pueden realizar.
En función de las variables mencionadas, llegamos a la siguiente clasificación:

\textbf{Cuadro 6: Caracterización de los Viajes a ser Relevados y Ubicación en el Formulario Base}

En la última columna del cuadro precedente se señalan dos situaciones: (a) en ciertos tipos de viajes (CDSV, VDSV, VR, VD), dentro de cada bloque, se indaga por la cantidad de viajes o visitas realizadas a cada lugar visitado y luego se realizan algunas preguntas sobre el último de los viajes o visitas a cada lugar; (b) en otros tipos de viajes (LDSV, VA, VE) se consulta directamente por cada uno de los viajes realizados.
Esto es así porque por la propia definición de los tipos de viaje que componen el primer grupo se trata de viajes muy parecidos entre sí; en cambio, en el segundo grupo, los viajes englobados dentro de un mismo tipo pueden ser totalmente distintos.

Excepto algún caso particular, existe un núcleo de preguntas que no varía según el tipo de viaje realizado por el hogar.
Por ejemplo, se consulta por destino, tipo de alojamiento, actividades, etc.

\hypertarget{secciuxf3n-ad-hoc-del-cuestionario}{%
\subsubsection{\texorpdfstring{\textbf{SECCIÓN ``AD-HOC'' DEL CUESTIONARIO}}{SECCIÓN ``AD-HOC'' DEL CUESTIONARIO}}\label{secciuxf3n-ad-hoc-del-cuestionario}}

Como fuera mencionado anteriormente, esta sección del cuestionario fue incorporada en la edición de 2013 de la EVyTH y tuvo como función primordial relevar las expectativas de viaje de los hogares argentinos durante fechas especiales dentro de cada mes de relevamiento, así como también, durante los periodos de alta demanda de turismo como el invierno y el verano.
La decisión de incorporar este relevamiento complementario se basó en la importancia de conocer anticipadamente (ex ante) los viajes que los hogares argentinos prevén realizar ante la implementación de una política pública como la incorporación y/o reestructuración de nuevos feriados nacionales por parte del gobierno nacional\footnote{De acuerdo con Decreto 245/2011 y la Ley 26.721.}, así como también para los periodos vacacionales como el periodo invernal y el periodo estival.
Por esta razón, al finalizar el cuestionario ``pétreo'', se incorpora una sección (o cuestionario ``ad-hoc'') sobre expectativas de viajes durante los períodos relevantes.

De acuerdo con El Pliego, la edición de la EVyTH (EVyTH-21) prevé el relevamiento de expectativas de viajes para las quincenas de temporada invernal (contemplando las expectativas de viajes que se realicen durante los meses de julio y agosto de 2021) y de verano (considerando las expectativas de viajes que se realizarán entre los meses de diciembre hasta marzo inclusive).

El relevamiento sobre expectativas para medir el caudal turístico y principales características del viaje en fechas especiales, se llevará a cabo durante quince días hábiles anteriores a la fecha de entrega del informe de resultados (siguiendo el Pliego, la entrega debe realizarse diez días antes del inicio de la temporada turística de invierno y verano).
En cada fecha clave, se relevará la cantidad de turistas, el destino (región turística para Argentina y ``Exterior'' para países extranjero), el motivo, el tipo de alojamiento y la duración del viaje para una muestra de 1.500 casos relevados.
Estas preguntas se realizan al final del bloque de preguntas sobre los miembros del hogar del cuestionario base y antes de la pregunta sobre ingresos del mismo cuestionario.
De esta manera, se mantiene la pregunta de menor respuesta, ingresos del hogar, en el último lugar para así obtener mayores posibilidades de concluir la encuesta en su totalidad.
Para su consulta ver sección 6.7).

\hypertarget{incorporaciones-sugeridas-al-formulario-base}{%
\subsection{\texorpdfstring{\textbf{INCORPORACIONES SUGERIDAS AL FORMULARIO BASE}}{INCORPORACIONES SUGERIDAS AL FORMULARIO BASE}}\label{incorporaciones-sugeridas-al-formulario-base}}

A continuación, se listan y comentan en términos generales algunas propuestas concretas que podrían incorporarse en el Formulario Base a lo largo del relevamiento a llevarse a cabo durante 2021.
La implementación eventual de estas propuestas queda sujeta al criterio del personal técnico del Organismo Contratante.

\hypertarget{uxedndice-de-confianza-del-turista}{%
\subsubsection{\texorpdfstring{\textbf{ÍNDICE DE CONFIANZA DEL TURISTA}}{ÍNDICE DE CONFIANZA DEL TURISTA}}\label{uxedndice-de-confianza-del-turista}}

Se propone implementar un nuevo módulo de preguntas destinado a la construcción de un Índice de Confianza del Turista de carácter continuo, y cuyos valiosos frutos podrán observarse cuando haya transcurrido al menos un año.
Deberá ser algo simple, que se realizará a todos los entrevistados al finalizar la encuesta base.
A continuación, una primera propuesta que deberá ser evaluada por el personal técnico del Organismo Contratante y testeada a través de una Prueba Piloto:

P1) En los próximos tres meses, ¿alguno de ustedes viajará por vacaciones, ocio, para visitar familiares o por otro motivo?
(no leer opciones, si dice ``no sé'' repreguntar si cree más probable que viajen o no)

\begin{enumerate}
\def\labelenumi{\arabic{enumi})}
\tightlist
\item
  Si (pase a P1C)
\item
  No (pase a P2)
\item
  No Sabe, aunque es más probable que viajen
\item
  No sabe, aunque es más probable que no viajen
\item
  No sabe, y no puede decir si es más probable que viajen o no. 9) No Responde (pase a P2)
\end{enumerate}

P2A) En todo el año 2021, ¿cree que ustedes viajarán por turismo más o menos que lo que han viajado en 2020?

\begin{enumerate}
\def\labelenumi{\arabic{enumi})}
\tightlist
\item
  Más
\item
  Menos
\item
  Lo mismo
\item
  NO VIAJAN POR TURISMO (pase a P3) 8) No sabe
\item
  No responde
\end{enumerate}

P2B) ¿Y considera que en el año 2021 podrán viajar más o menos que lo que viajarán durante este año?

\begin{enumerate}
\def\labelenumi{\arabic{enumi})}
\tightlist
\item
  Más
\item
  Menos
\item
  Lo mismo
\item
  No sabe
\item
  No responde
\end{enumerate}

\hypertarget{indagaciuxf3n-sobre-los-hogares-que-se-incorporan-al-turismo}{%
\subsubsection{\texorpdfstring{\textbf{INDAGACIÓN SOBRE LOS HOGARES QUE SE INCORPORAN AL TURISMO}}{INDAGACIÓN SOBRE LOS HOGARES QUE SE INCORPORAN AL TURISMO}}\label{indagaciuxf3n-sobre-los-hogares-que-se-incorporan-al-turismo}}

La comparación de los resultados obtenidos en las ediciones anteriores de esta encuesta da cuenta de un incremento en la proporción de personas y hogares que realizan viajes turísticos a partir de 2006.
En este escenario cabría la posibilidad de indagar sobre cuál es la proporción de hogares que están viajando por primera vez.
El abanico de formas metodológicas para mensurar esta cuestión es amplio y complejo a la vez, lo que ameritaría discusiones profundas sobre la forma de implementarlo, así como los tipos de viajes turísticos a considerar (cualquier tipo o sólo los de ocio, por ejemplo).
Por ejemplo, podría preguntarse a los hogares si es la primera vez que visitan ese destino.

\hypertarget{motivo-congresos-para-aquellos-que-viajan-por-trabajo}{%
\subsubsection{\texorpdfstring{\textbf{MOTIVO CONGRESOS PARA AQUELLOS QUE VIAJAN POR TRABAJO}}{MOTIVO CONGRESOS PARA AQUELLOS QUE VIAJAN POR TRABAJO}}\label{motivo-congresos-para-aquellos-que-viajan-por-trabajo}}

Algunas ciudades se caracterizan por su particular atracción como destinos para congresos y convenciones.
El cuestionario actual no permite identificar cuántos de los viajes que son realizados como motivo de trabajo se corresponden con viajes para asistir a congresos/convenciones por trabajo.
Sería posible incorporar una pregunta que saldara esta deuda.

\hypertarget{profundizaciuxf3n-de-los-aspectos-referidos-a-los-viajes-al-exterior-del-pauxeds-por-parte-de-los-residentes-que-no-estuxe1n-siendo-captados-por-la-eti}{%
\subsubsection{\texorpdfstring{\textbf{PROFUNDIZACIÓN DE LOS ASPECTOS REFERIDOS A LOS VIAJES AL EXTERIOR DEL PAÍS POR PARTE DE LOS RESIDENTES QUE NO ESTÁN SIENDO CAPTADOS POR LA ETI}}{PROFUNDIZACIÓN DE LOS ASPECTOS REFERIDOS A LOS VIAJES AL EXTERIOR DEL PAÍS POR PARTE DE LOS RESIDENTES QUE NO ESTÁN SIENDO CAPTADOS POR LA ETI}}\label{profundizaciuxf3n-de-los-aspectos-referidos-a-los-viajes-al-exterior-del-pauxeds-por-parte-de-los-residentes-que-no-estuxe1n-siendo-captados-por-la-eti}}

Actualmente, la Encuesta de Turismo Internacional releva los viajes de los residentes al exterior que se realizan desde los principales aeropuertos del país (Ezeiza, Aeroparque y Córdoba) y desde la terminal portuaria de la Ciudad de Buenos Aires.
Por tal motivo, una parte sustancial de los viajes que los residentes realizan hacia países limítrofes por otros medios de locomoción (automóvil y ómnibus, particularmente) no son captados por dicho relevamiento.
Una forma posible de subsanar esta laguna de información sería agregar algunas breves preguntas que permitan conocer las principales características de estos viajes, además de las que se relevan habitualmente para todos los viajes al exterior.
Si bien por razones de fiabilidad estadística los resultados obtenidos para un mes no serán suficientes, la acumulación luego de una serie de ondas (semestral, anual) permitirá contar con una cantidad de información sustancial para realizar análisis exploratorios.

\hypertarget{propensiuxf3n-al-turismo-de-los-hogares}{%
\subsubsection{\texorpdfstring{\textbf{PROPENSIÓN AL TURISMO DE LOS HOGARES}}{PROPENSIÓN AL TURISMO DE LOS HOGARES}}\label{propensiuxf3n-al-turismo-de-los-hogares}}

Podría resultar de interés tener alguna apreciación sobre las actividades turísticas del hogar en un marco temporal que exceda los meses específicos sobre los que se indaga acerca de la realización de viajes y se acerque más a captar la importancia que el turismo tiene para ese hogar.
Por ejemplo, podría indagarse acerca de si habitualmente suelen tomarse vacaciones en la temporada de verano, si viajan en algún o algunos feriados turísticos y si además realizan viajes fuera de estas fechas.
En consonancia con esto, cabría indagar, por ejemplo, con cuanto tiempo de anticipación planean sus vacaciones estivales, y desde cuando comienzan a ahorrar para tal fin.
Este módulo podría relevarse en los meses de temporada baja.

\hypertarget{etapas-del-viaje}{%
\subsubsection{\texorpdfstring{\textbf{ETAPAS DEL VIAJE}}{ETAPAS DEL VIAJE}}\label{etapas-del-viaje}}

Una de las preguntas planteadas en el Formulario Base se refiere a cuántos lugares visitaron además del destino principal.
Una posible implementación sería que se indague sobre cuáles fueron las provincias y localidades visitadas, en el caso de que se realicen viajes al exterior cuales fueron los países respectivamente.

\hypertarget{utilizaciuxf3n-de-internet}{%
\subsubsection{\texorpdfstring{\textbf{UTILIZACIÓN DE INTERNET}}{UTILIZACIÓN DE INTERNET}}\label{utilizaciuxf3n-de-internet}}

Se sugiere modificar la pregunta actual sobre la utilización de Internet para consultar o contratar información relacionada al viaje.
Se propone indagar si la contratación de alojamiento pago, medio de transporte público y paquete fue por medio de Internet o no.
Y, en caso de no haber adquirido ese tipo de servicios y/o las respuestas resultaran negativas, se realizaría una pregunta genérica sobre el uso de internet, por ejemplo: ``¿USTED O ALGUNA DE LAS PERSONAS QUE PARTICIPÓ DEL VIAJE, USARON INTERNET PARA ORGANIZAR EL VIAJE O CONSULTAR INFORMACIÓN SOBRE CÓMO VIAJAR, DÓNDE COMER, QUÉ HACER, EL CLIMA, ETC.?''.

\emph{Nota 1}: se podría distinguir entre contratación en línea vía OTA (Booking, Despegar, Plataforma 10, etc.) y gestión/reserva por internet distinto de OTA (por ej., contratar una cabaña por haber visto en la web y en base a eso contactar al dueño directo).

\emph{Nota 2}: Se haría necesario definir si se continúa indagando esto sólo para los viajes por ocio no a segunda vivienda o si se extendería a todos los viajes (incluso, también a las visitas de un día).

\hypertarget{parte-interna-de-viajes-emisivos}{%
\subsubsection{\texorpdfstring{\textbf{PARTE INTERNA DE VIAJES EMISIVOS}}{PARTE INTERNA DE VIAJES EMISIVOS}}\label{parte-interna-de-viajes-emisivos}}

Se sugiere evaluar la factibilidad de incluir pregunta simple que capte la existencia o no de un destino interno en viajes emisivos.
Por ejemplo: ``¿SU VIAJE AL EXTERIOR INCLUYÓ AL MENOS UNA NOCHE ALOJADOS EN ALGUNA LOCALIDAD ARGENTINA?''.

\hypertarget{evaluaciuxf3n-de-destinos-alternativos}{%
\subsubsection{\texorpdfstring{\textbf{EVALUACIÓN DE DESTINOS ALTERNATIVOS}}{EVALUACIÓN DE DESTINOS ALTERNATIVOS}}\label{evaluaciuxf3n-de-destinos-alternativos}}

Se propone reemplazar el enunciado actual por una redacción más coloquial, a fines de obtener un mejor nivel de respuesta.
Por ejemplo: ``¿DESDE EL INICIO HABÍAN DEFINIDO VIAJAR A ESTE LUGAR, O TAMBIÉN PENSARON EN OTROS LUGARES ANTES DE DECIDIRSE A VISITAR\ldots..?''.
Las opciones de respuesta serían: 1-Destino decidido desde el inicio; 2- Pensaron en otros lugares.

\hypertarget{desagregaciuxf3n-del-gasto-de-viajes-y-visitas}{%
\subsubsection{\texorpdfstring{\textbf{DESAGREGACIÓN DEL GASTO DE VIAJES Y VISITAS}}{DESAGREGACIÓN DEL GASTO DE VIAJES Y VISITAS}}\label{desagregaciuxf3n-del-gasto-de-viajes-y-visitas}}

Como propuesta a implementar en el Formulario Base, se puede considerar la desagregación del gasto.
En el cuestionario implementado en la EVyTH-21, una de las preguntas referidas al gasto era la siguiente: ``Aproximadamente, ¿Cuánto gastaron en total en el viaje, incluyendo los gatos de todos los miembros del hogar en transporte, alojamiento, comidas, excursiones, etc.?''; se sugiere incorporar preguntas para la apertura del mismo gasto total (que continuaría indagándose, al inicio para que no se vea afectado el nivel de respuesta por las preguntas desglosadas).
A continuación, se detalle el desglose que se sugiere realizar:

\begin{enumerate}
\def\labelenumi{\arabic{enumi})}
\tightlist
\item
  Transporte: incluye gastos en servicios de transporte público (micro, ómnibus, tren, avión, etc.) como nafta y peajes. En todos los casos, refiere al transporte entre origen-destino (y, eventualmente, entre etapas del viaje). Esto es primordial pues permite discriminar el gasto en destino.
\item
  Alojamiento: en todo tipo de alojamientos por los que se deba pagar (hoteles y similares, camping, viviendas de alquiler temporario, otros pagos).
\item
  Resto: Bajo esta variable se reunirían los gastos que no se incluyen en los ítems anteriores, como ser alimentación, excursiones, entradas a espectáculos, regalos y suvenires, etc. Sólo para los viajes y visitas internos.
\end{enumerate}

\hypertarget{relevamientos-ad-hoc-informes-no-buxe1sicos}{%
\subsection{\texorpdfstring{\textbf{RELEVAMIENTOS AD HOC: INFORMES NO BÁSICOS}}{RELEVAMIENTOS AD HOC: INFORMES NO BÁSICOS}}\label{relevamientos-ad-hoc-informes-no-buxe1sicos}}

A partir de la reunión de capacitación con el personal técnico del Organismo Contratante realizada con motivo de dar cierre a la EVyTH-17, surge la necesidad de replantear algunos ejes temáticos de los informes ``ad hoc'', incorporados a la Encuesta durante las pasadas experiencias de la EVyTH.
La lista de nuevas variables definitivas a incorporar en el relevamiento debe estar sujeta a las necesidades que plantee el equipo técnico del Organismo Contratante, sin embargo, a continuación, se enumeran posibles temáticas a incorporar:

\begin{enumerate}
\def\labelenumi{\arabic{enumi}.}
\item
  Destino Ideal: Se podría incorporar, por ejemplo, en los relevamientos de expectativas de invierno y verano, una pregunta por el destino ideal de los hogares suponiendo que el hogar dispusiera de los recursos necesarios (tiempo, dinero, etc.).
  Se replica experiencia pasada, si bien los resultados no revisten mayor interés (pues no variarán mucho de lo que se conoce), el Organismo Contratante los puede encontrar positivos para su comunicación.
\item
  Conocimiento y Deseo de Conocer Destinos Turísticos Estratégicos del País: Mediante la implementación de un módulo ad hoc, dirigido sólo a la persona que responde la encuesta (no a la totalidad de miembros del hogar) podría indagarse acerca del conocimiento o no, y sobre las ganas de visitar por primera vez o de regresar a destinos turísticos estratégicos de Argentina (Cataratas del Iguazú, Bariloche, Quebrada de Humahuaca, Sierras de Córdoba, Glaciar Perito Moreno, por ejemplo).
  La respuesta del entrevistado, para cada lugar, quedaría englobada en alguna de las siguientes categorías: No lo conoce y tiene interés en visitarlo, No lo conoce y no tiene interés en visitarlo, lo conoce y le gustaría visitarlo nuevamente, lo conoce, pero no volvería.
  Al término de un periodo temporal amplio (semestral, por ejemplo) se tendría una muestra lo suficientemente significativa para analizar esta información no sólo para el total de los aglomerados sino también de acuerdo a las regiones y a diferentes características de las personas.
  Por razones referidas a la duración de la entrevista, este módulo podría incorporarse en los meses en que se releva información sobre viajes turísticos realizados en los meses de temporada baja.
  Misma lógica podría seguirse para destinos ``estratégicos'', como Iberá.
\item
  Estudio de los hábitos turísticos de la población: amén de si viajan o no, clasificar a los hogares viajeros según frecuencia, tipos de viaje (uno largo anual, varios cortos, etc.), destinos (cercanos/lejanos; Argentina/Exterior), trayectorias (si siempre viajaron, si lo empezaron a hacer en los últimos años, etc.), etc.
\item
  Incorporación ``temporal'' de preguntas en cuestionario base: esta es otra modalidad de relevamiento ad hoc, y consistiría en profundizar por un determinado tiempo (vinculado a la precisión requerida/tamaño de muestra mínima) sobre algún aspecto del viaje.
  Por ejemplo, un breve módulo sobre la importancia de la gastronomía en viajes internos, profundizar sobre algunas características de los viajes emisivos para aumentar la comprensión sobre todo de los ``no ETI'', etc.
\end{enumerate}

Por otra parte, algunos informes no periódicos no requieren ser realizados una vez al año (por ejemplo, el informe de Destinos) ya que presentan un gran componente estructural y, al mismo tiempo, realizar este informe con información agregada de al menos 3 años permitirá robustecer los resultados al agregar muestra durante un período más prologando.
Por esto se sugiere realizar algunos cambios:

\begin{enumerate}
\def\labelenumi{\arabic{enumi}.}
\item
  Comportamiento turístico: Se sugiere incorporar información sobre destino en Argentina/Exterior.
\item
  Destinos: Se sugiere realizar este informe con información agregada 2012-2020 (8 años completos) y modificar el nombre a ``Eje PFETS''.
  Además, se sugiere revisar otras clasificaciones (tipo de atractivo, demográfico, oferta).
\end{enumerate}

\hypertarget{consistencia-y-validaciuxf3n}{%
\section{\texorpdfstring{\textbf{CONSISTENCIA Y VALIDACIÓN}}{CONSISTENCIA Y VALIDACIÓN}}\label{consistencia-y-validaciuxf3n}}

Con el objeto de garantizar la calidad de la información se utilizará, tal como se expuso en La Oferta, un proceso de validación y consistencia de los ficheros resultantes de la encuesta para aplicarse previo a cualquier imputación o tratamiento de información faltante. Dicho proceso se implementará mediante un programa que permita gestionar el proceso de validación y las correspondientes correcciones en los casos de hallar inconsistencias que ha sido diseñado especialmente para la EVyTH (este sistema ya ha sido probado con éxito en ediciones anteriores de la EVyTH --como, por ejemplo, en la EVyTH-11, EVyTH-12, EVyTH- 13, EVyTH-14, EVyTH-15, EVyTH-16, EVyTH-17, EVyTH-18, EVyTH-19 y la EVyTH-20).
En este tipo de procesos, el objetivo es controlar los valores o el campo de variación (rango de valores) de cada variable del cuestionario. En una primera etapa la validación se realiza exclusivamente para cada uno de los individuos registrados, independientemente de su interrelación con la información relevada en otros individuos del mismo hogar. Esto significa que esta validación inicial se realiza en forma independiente para cada individuo y para cada variable del cuestionario.
En las preguntas cerradas se controla que los códigos introducidos se encuentren dentro del rango de valores correspondientes a dicha variable. Por ejemplo, en la variable ``NIVEL DE INSTRUCCIÓN'', sólo deben presentarse los códigos 01 a 10 y el código 99 para el caso de no respuesta.
Una segunda etapa del proceso de validación y consistencia consiste en la validación cruzada de la información contenida entre variables relevadas en un mismo individuo. Es decir, se analizan las interrelaciones lógicas entre dos o más preguntas de un mismo individuo. Por ejemplo, si no responde en la pregunta sobre si trabaja o no, pero luego responde que es ``ama de casa'', entonces debería corresponder la respuesta ``No'' a la pregunta sobre si trabaja.

\hypertarget{consideraciones-finales}{%
\chapter*{Consideraciones Finales}\label{consideraciones-finales}}
\addcontentsline{toc}{chapter}{Consideraciones Finales}

  \bibliography{book.bib,packages.bib}

\end{document}
